\subsection{\name{YuvKA.VideoModel}}

\paragraph{\name{Rgb}}

\begin{itemize}

\item{\name{TestEquality}}~\\
Dieser Test überprüft, ob die Überschreibung der Operatoren bzw. Methoden \verb#==#, \verb#!=# und \verb#Equals# in dem Rgb-Struct korrekt funktioniert.

\item{\name{TestHashing}}~\\
Dieser Test überprüft, ob die Überschreibung der Methode \verb#GetHashCode# in dem Rgb-Struct korrekt funktioniert.

\item{\name{TestToString}}~\\
Dieser Test überprüft, ob die Überschreibung der Methode \verb#ToString# in dem Rgb-Struct korrekt funktioniert.
\end{itemize}

\subsection{\name{YuvKA.Pipeline}}

\paragraph{\name{PipelineGraph}}
\begin{itemize}
	\item \name{DfsFindsCorrectNodes} \\
	Stellt sicher, dass der DFS-Algorithmus im allgemeinen Fall korrekt implementiert wurde.

	\item \name{PipelineGraphCanDetectCycles} \\
	Stellt sicher, dass Zyklen korrekt erkannt werden und DFS auch hier funktioniert.

	\item \name{PipelineGraphCanHandleSplits} \\
	Stellt sicher, dass das Spalten und Wiederzusammenführen von Pipelines im DFS korrekt behandelt wird.

	\item \name{TestAddNode} \\
	Stellt sicher, dass beim Hinzufügen von Knoten deren Namen korrekt gesetzt werden.

	\item \name{TestRemoveNode} \\
	Stellt sicher, dass beim Entfernen eines Knotens aus der Pipeline alle zugehörigen Kanten entfernt werden.

	\item \name{TestReturnNumberOfFramesToPrecompute} \\
	Stellt sicher, dass die Anzahl der vorzuberechnenden Frames korrekt berechnet wird.
\end{itemize}

\paragraph{\name{PipelineDriver}}
\begin{itemize}
	\item \name{RenderTicksClosesObservable} \\
	Stellt sicher, dass ein asynchroner Abbruch des Renderns den ausgegbenen Videostream auch tatsächlich beendet.
    \item \name{RenderTicksEarlyCancellation} \\
    Stellt sicher, dass nach einem asynchronen Abbruch kein weiterer Tick zu rendern begonnen wird.
    \item \name{RenderTicksNonReentrant} \\
    Stellt sicher, dass die Process-Methode für jeden Knoten nicht zweimal gleichzeitig betreten wird.
    \item \name{RenderTicksProcessesDiamondGraph} \\
    Stellt sicher, dass ein Graph mit Verzweigung und folgender Zusammenführung korrekt gerendert wird.
    \item \name{RenderTicksProcessesLongGraph} \\
    Stellt sicher, dass ein linearer Graph mit 100 Knoten korrekt gerendert wird.
    \item \name{RenderTicksPropagatesExceptions} \\
    Stellt sicher, dass Exceptions während des Renderns nicht verschluckt werden.
    \item \name{RenderTicksStressTest} \\
    Führt \name{RenderTicksLongGraph} 100-mal hintereinander aus.
\end{itemize}

\paragraph{Manipulationsknoten}

\begin{itemize}

\item{\name{TestAdditiveMerge}}~\\
Dieser Test überprüft, ob der \name{AdditiveMergeNode} die Farben der Pixel der Eingabeframes korrekt addiert.

\item{\name{TestDelayNode}}~\\
Dieser Test überprüft, ob der \name{DelayNode} die Eingabeframe korrekt verzögert.

\item{\name{TestDifferenceNode}}~\\
Dieser Test überprüft, ob der \name{DifferenceNode} die Farben der Pixel der Eingabeframes korrekt voneinander abzieht.

\item{\name{TestGaussianBlurMonocolor}}~\\
Dieser Test überprüft, ob der \name{BlurNode} mit gaußschem Blur und einer einfarbigen Frame keinerlei Effekt besitzt.

\item{\name{TestGaussianZeroBlur}}~\\
Dieser Test überprüft, ob der \name{BlurNode} mit gaußschem Blur und einem Radius von 0 keinerlei Effekt auf die Frame besitzt.

\item{\name{TestLinearBlurMonocolor}}~\\
Dieser Test überprüft, ob der \name{BlurNode} mit linearem Blur und einer einfarbigen Frame keinerlei Effekt besitzt.

\item{\name{TestLinearZeroBlur}}~\\
Dieser Test überprüft, ob der \name{BlurNode} mit linearem Blur und einem Radius von 0 keinerlei Effekt auf die Frame besitzt.

\item{\name{TestInverter}}~\\
Dieser Test überprüft, ob der \name{InverterNode} die Farben der Pixel der Eingabeframe korrekt invertiert.

\item{\name{TestRgbSplit}}~\\
Dieser Test überprüft, ob der \name{RgbSplitNode} die Farben der Pixel der Eingabeframe korrekt in die drei Farbkanäle aufspaltet.

\item{\name{TestWeightedAverageMerge}}~\\
Dieser Test überprüft, ob der \name{WeightedAverageMerge} die Farben der Pixel der Eingabeframes korrekt bezüglich ihrer Gewichtung addiert.

\end{itemize}

\paragraph{AusgabeKnoten}

\begin{itemize}

\item{\name{TestArtifactOverlay}}~\\
Dieser Test überprüft die Überlagerung von Artefakten, indem zwei Beispielframes erstellt werden und das Ergebnis der Artefaktüberlagerung zu einer manuellen Betrachtung als Bilddatei gespeichert wird.

\item{\name{TestMacroBlockOverlay}}~\\
Dieser Test überprüft die Überlagerung von Makroblockpartitionsentcheidungen, indem alle mögliche Entscheidungen auf eine Testframe gezeichnet werden und diese zur manuellen Betrachtung als Bilddatei gespeichert wird.

\item{\name{TestNoOverlay}}~\\
Dieser Test überprüft ob die Überlagerung durch \name{NoOverlay} tatsächlich keinerlei Effekt auf die Eingabeframe besitzt.

\item{\name{TestOverlayNode}}~\\
Dieser Test überprüft ob der \name{OverlayNode}, im speziellen die Property \verb#InputIsValid#, korrekt funktioniert.

\item{\name{TestVecorOverlay}}~\\
Dieser Test überprüft die Überlagerung von Bewegungsvektoren pro Makroblock, indem es verschiedenste Bewegungsvektoren auf eine Testframe zeichnet und diese zur manuellen Betrachtung als Bilddatei gespeichert wird.

\end{itemize}

\paragraph{\name{DiagramNode}}
\begin{itemize}
	\item{\name{TestDiagramNode}} \\
		Testet die allgemeine Funktion des DiagramNodes.
	\item{\name{TestNoDrawingIfDiagramNodeNotEnabled}} \\
		Stellt sicher, dass der DiagramNode nicht arbeitet, wenn er deaktiviert ist.
	
	\item{\name{RedrawOnTickSetBack}} \\
		Stellt sicher, dass der DiagramNode die aktuellen Werte seiner Graphen löscht, sobald der Tick vor den aktuellen Zeitpunkt gesetzt wird.
\end{itemize}


\paragraph{\name{HistogramNode}}
\begin{itemize}
	\item{\name{TestHistogramNodeRGB}} \\
		Testet die allgemeine Funktion der RGB-Methoden des HistogramNodes.
	
	\item{\name{TestHistogramNodeValue}} \\
		Testet die allgemeine Funktion der Value\footnote{im HSV-Farbraum}-Methode des HistogramNodes.
\end{itemize}

\subsection{\name{YuvKA.ViewModel}}

\paragraph{\name{MainViewModel}}
\begin{itemize}
	\item \name{CanOpenCloseOutputWindow} \\
	Stellt sicher, dass das View Model nach Öffnen und Schließen eines Ausgabefenster jeweils den korrekten Zustand annimmt.
	\item \name{OpenCreatesNodeViewModelsFromModel} \\
	Stellt sicher, dass die Pipeline in Model und View Model nach Sichern und Laden korrekt wiederhergestellt wurde.
	\item \name{ToolboxCanHasBlurNode} \\
	Stellt sicher, dass die Plugin-Knoten korrekt in die Toolbox geladen werden.
	\item \name{UndoRedoWorks} \\
	Überprüft den Undo-/Redo-Mechanismus anhand einer festgelegten Abfolge von Kommandos.
\end{itemize}

\paragraph{\name{OverlayViewModel}}

\begin{itemize}

\item{\name{TestGeneralFunctions}}~\\
Dieser Test überprüft die Funktionalitäten von \name{OverlayViewModel}, im speziellen die Properties \verb#TypeTuples# und \verb#ChosenType# sowie die Methode \verb#Handle(TickRenderedMessage message)#, auf Korrektheit.

\end{itemize}

\paragraph{\name{PipelineViewModel}}
\begin{itemize}
	\item \name{CanAddInput} \\
	Stellt sicher, dass beim Droppen einer Kante auf einen Knoten mit dynamischer Inputzahl ein neuer Input hinzugefügt wird.
	\item \name{CanDragEdge} \\
	Stellt sicher, dass eine Kante beim Ziehen jeweils den korrekten Zustand und die korrekte Position annimmt.
	\item \name{CanDropNode} \\
	Stellt sicher, dass ein Knoten beim Drop auf die Pipeline korrekt erstellt und hinzugefügt wird.
	\item \name{CanMoveNode} \\
	Stellt sicher, dass ein Knoten beim Ziehen jeweils die richtige Position annimmt.
	\item \name{CanRemoveNode} \\
	Stellt sicher, dass sich Model und View Model nach Löschen eines Knotens im korrekten Zustand befinden.
	\item \name{DisplaysDynamicallyAddedOutputs} \\
	Stellt sicher, dass sich das View Model nach Hinzufügen eines Outputs im Model im korrekten Zustand befindet.
	\item \name{NotifiesOfViewPositionChanges} \\
	Stellt sicher, dass das Setzen der Position eines Knotens die \name{ViewPositionChanged}-Observable triggert.
	\item \name{RendersNodeToFileCorrectly} \\
	Stellt sicher, dass der Output eines Knotens korrekt gerendert und gespeichert wird.
\end{itemize}

\paragraph{\name{ReplayStateViewModel}}
\begin{itemize}
	\item \name{CantPlayInvalidPipeline} \\
	Stellt sicher, dass eine ungültige Pipeline nicht abgespielt werden kann.
	\item \name{PlayPauseWorks} \\
	Stellt sicher, dass ein Aufruf von \name{PlayPause} zum Starten des PipelineDrivers und ein zweiter Aufruf zu dessen Abbruch führt.
	\item \name{StopsAtEndThenStartsAnew} \\
	Stellt sicher, dass die Wiedergabe am Ende des Videos automatisch stoppt und ein Aufruf von \name{PlayPause} das Video wieder vom Anfang an abspielt.
\end{itemize}

\subsection{\name{YuvKA.ViewModel.PropertyEditor}}

\paragraph{\name{PropertyEditorViewModel}}

\begin{itemize}

\item{\name{TestPEVM}}~\\
Dieser Test überprüft, ob das \name{PropertyEditorViewModel} die Properties des Objektes auf korrekte PropertyViewModels delegiert.

\end{itemize}

\paragraph{\name{PropertyViewModel}}

\begin{itemize}

\item{\name{GeneralPropertyViewModelTest}}
Dieser Test überprüft die allgemeinen Funktionen des \name{PropertyViewModel}s, speziell \verb#DisplayName#, auf Korrektheit.

\end{itemize}

\paragraph{\name{DiagramViewModel}}
\begin{itemize}
	\item{\name{CanAddLine}} \\
		Stellt sicher, dass das DiagramViewModel seinem Diagramm Kurven hinzufügen kann.

	\item{\name{CanDeleteLine}} \\
		Stellt sicher, dass das DiagramViewModel Kurven von seinem Diagramm löschen kann.

	\item{\name{GetsData}} \\
		Stellt sicher, dass die berechneten Daten dem DiagramViewModel korrekt übergeben werden.
	
	\item{\name{ShowOnlyAvailableTypes}} \\
		Stellt sicher, dass nur diejenigen Graphentypen angezeigt werden, die mit den aktuellen Einstellungen möglich sind.
	
	\item{\name{ChangeChosenType}} \\
		Stellt sicher, dass der verwendete Graphentyp vom Benutzer geändert werden kann.
	
	\item{\name{ResetRefWithExistingGraphControls}} \\
		Stellt sicher, dass die Grapheneinstellungen aktualisiert werden, wenn das Referenzvideo geändert wird.
	
	\item{\name{CannotAddGcWithoutChosenVid}} \\
		Stellt sicher, dass keine Grapheneinstellungen hinzugefügt werden kann, wenn kein Video ausgewählt ist.
\end{itemize}


\paragraph{\name{HistogramViewModel}}
\begin{itemize}
	\item{\name{GetsData}} \\
		Stellt sicher, dass die berechneten Daten dem HistogramViewModel korrekt übergeben werden.
\end{itemize}
