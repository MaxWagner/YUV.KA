\paragraph{\name{/T10/ Pipeline-Konstruktion}}
\begin{itemize}
	\item Der Benutzer startet das Programm und wählt ``Neue Pipeline erstellen'', bzw. drückt auf den ``New'' Button.
	\item Der Benutzer erstellt jeden Knoten ein Mal per ``Drag-and-Drop''
	\item Er verbindet die erstellten Knoten durch Kanten
	
	In diesem automatischen Test wird jeder Knoten genau ein Mal erstellt und anschließend werden diese in einer beliebigen Reihenfolge durch Kanten verbunden.
\end{itemize}

\paragraph{\name{/T20/ Manipulation einer Pipeline}} ~\\

\begin{itemize}
	\item Der Benutzer startet das Programm und erstellt eine beliebige Pipeline mit mindestens zwei Knoten, davon mindestens ein Manipulationsknoten mit Optionen und mindestens einer Kante.
			\item Der Benutzer bewegt einen gesetzten Knoten innerhalb der Pipeline per ``Drag-and-Drop''.
			\item Der Benutzer verändert das Ziel einer gesetzten Kante per ``Drag-and-Drop''.
			\item Der Benutzer löscht eine Kante.
			\item Der Benutzer verändert eine Einstellung eines Manipulationsknotens.
			\item Der Benutzer löscht einen Knoten.
\end{itemize}

Wird durch einen automatischen Test abgedeckt, wobei die Pipeline aus drei Weichzeichnungsknoten besteht. Das Ändern einer Einstellung kann nicht vollkommen durch einen automatischen Test simuliert werden. Deswegen geschieht dies nicht auf dem ViewModel, sondern direkt auf dem zugrundeliegendem Knotenmodell.

\paragraph{\name{/T30/ Sicherung einer Pipeline}} ~\\

\begin{itemize}
	\item Der Benutzer startet das Programm und erstellt eine beliebige Pipeline.
	\item Der Benutzer speichert die Pipeline.
	\item Der Benutzer wählt ``Neue Pipeline erstellen''.
	\item Der Benutzer lädt die gesicherte Pipeline.
\end{itemize}

Wird durch einen automatischen Test abgedeckt, wobei der Dialog, der beim Abspeichern und Laden der Pipeline aufgerufen wird, nicht vollkommen durch einen automatisierten Test simuliert werden kann.

\paragraph{\name{/T40/ Videoverarbeitung}}

\paragraph{\name{/T50/ Videoanalyse}} ~\\
 
\begin{itemize}
	\item Der Benutzer startet das Programm und erstellt eine beliebige, zusammenhängende, zyklenfreie Pipeline mit mindestens einem Eingabe-, Überlagerungs-, sowie Diagrammknoten.
	\item Der Benutzer deaktiviert einen Diagrammknoten.
	\item Der Benutzer öffnet einen Überlagerungsknoten.
	\item Der Benutzer beginnt die Videowiedergabe.
	\item Der Benutzer fügt eine Überlagerungsoption hinzu, während das Video abgespielt wird.
	\item Der Benutzer entfernt die hinzugefügte Überlagerungsoption.
	\item Der Benutzer setzt die Videowiedergabe zurück und schließt den Überlagerungsknoten.
	\item Der Benutzer reaktiviert den deaktivierten Diagrammknoten und öffnet diesen.
	\item Der Benutzer wählt ein Referenzvideo aus und fügt einen Analysegraphen hinzu.
	\item Der Benutzer startet erneut die Videowiedergabe.
	\item Der Benutzer ändert den Typ des Analysegraphen.
	\item Der Benutzer löscht den Analysegraphen.
\end{itemize}

Auch dieser Test wird durch eine automatische Version abgedeckt. Der Test dient dem Abtesten der grundlegenden Funktion von Überlagerungs- und Diagrammknoten.
