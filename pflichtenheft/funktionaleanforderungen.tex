\section{Funktionale Anforderungen}

\subsection{Input/Output zur Videobearbeitung}
 
\begin{description}
        \item[/F10/]YUV-Dateien einlesen \label{F10}\newline
                Als YUV-Datei gespeicherte Videos sollen eingelesen und anschließend bearbeitet werden können.
        \item[/F20/]PNG-Dateien als Standbildvideo \label{F20}\newline
                PNG-Dateien sollen eingelesen und als Standbild abgespielt werden können. Der Benutzer erhält hiermit ein Video, das nur das eingelesene Bild als Standbild zeigt.
        \item[/F30/]Konstante Farbe als Input \newline
                Der Benutzer soll in der Lage sein auf einfache Weise eine beliebige Farbe als Standbildinput auszuwählen.
        \item[/F40/]Perlin-Noise \newline
                Der Benutzer soll Perlin-Noise als Videoinput erzeugen können.
        \item[/F50/]Speichern von YUV-Dateien \newline
                Eventuell veränderte Videos sollen im YUV-Format gespeichert werden können.
        \item[/F60/]Speichern von Pipelines \label{F60}\newline
                Der Benutzer soll die erstellten Bearbeitungs- und Analysepipelines speichern und laden können.
        \item[/F70/]Anzeigen von (bearbeiteten) Videos
\end{description}
 
\subsection{Videobearbeitung}
\begin{description}
        \item[/F100/]RGB-Trennung \newline
                 Der Benutzer soll ein Video so auftrennen können, dass er drei Videos erhält, von denen das Erste nur den roten Anteil, das Zweite nur den grünen Anteil und das Dritte nur den blauen Anteil des aufgetrennten Videos zeigt.
        \item[/F110/]Additives Übereinanderlegen von Videos \newline
                Mehrere Videos sollen durch Addition ihrer Pixelwerte übereinandergelegt werden können. Hierbei werden die Farbwerte der positionsgleichen Pixel der Videos aufaddiert und als neuer Farbwert des entstehenden Videos interpretiert.
        \item[/F120/]Gewichtet gemitteltes Übereinanderlegen von  Videos \newline
                Der Benutzer soll in der Lage sein mehrere Videos gemittelt übereinander zu legen. Hierbei werden die Farbwerte positionsgleicher Pixel gewichtet aufaddiert und anschließend gemittelt, um im gegebenen Farbbereich zu bleiben. Der Benutzer ist außerdem in der Lage die Gewichtung selbst festzulegen.
        \item[/F130/]Lineares und Gauß'sches Weichzeichnen \newline
                Lineares und Gauß'sches Weichzeichnen von Videos soll möglich sein.
        \item[/F140/]Farbinvertierung \newline
                Eine Farbinvertierung der Videos soll möglich sein.
        \item[/F150/]Kontrast, Farbsättigung und Helligkeit editieren \newline
                Der Benutzer soll in der Lage sein den Kontrast, die Farbsättigung und die Helligkeit der Videos zu verändern.
        \item[/F160/]Verzögern eines Videos um $n$ Frames \newline
                Der Benutzer soll ein Video um $n$ Frames nach hinten versetzen können.       
        \item[/F170/]Bilden des Differenzvideos zweier Videos \newline
                Es soll möglich sein das Differenzvideo durch Subtraktion der Farbwerte positionsgleicher Pixel der beiden Videos zu erstellen.
	\item[/F180/]Verwendung eines Videoencoders \newline
		Der Benutzer soll in der Lage sein den zu testenden Videoencoder in die Pipeline einzubauen.
\end{description}
 
\subsection{Videoanalyse}
\subparagraph{Analyse eines einzelnen Videos} 
	\begin{description}
	        \item[/F200/]Anzeigen des Inter/Intra-Block-Verhältnisses pro Frame
                \item[/F210/]Anzeigen des Helligkeitsverlaufs eines Videos pro Frame
                \item[/F220/]Anzeigen des Histogramms eines Video
                \item[/F230/]Anzeigen der ``peak signal-to-noise ratio pro Frame''
        \end{description}
\subparagraph{Vergleich eines oder mehrerer Videos mit einem Referenzvideo}
        \begin{description}
   	        \item[/F300/]Anzeigen der Differenzen der Pixelfarben
                \item[/F310/]Anzeigen der Differenz der Encoder-Entscheidungen pro Frame
                \item[/F320/]Anzeigen der Anzahl der Artefakte pro Frame
                \item[/F330/]Videoansicht mit Artefakthighlighting \newline
			Der Benutzer soll sich eine Ansicht anzeigen lassen können, die ihm die Artefakte im Bezug auf das Referenzvideo hervorhebt.
        \end{description}  
\subsection{Sonstiges}
	\begin{description}
		\item[/F400/]Im Video springen \newline
			Der Benutzer soll sowohl bei laufendem, als auch bei nicht laufendem Video an einen beliebigen Punkt im Video springen können.
		\item[/F410/]Video schneller/langsamer abspielen \newline
			Der Benutzer soll in der Lage sein das Video auf doppelter und halber Geschwindigkeit abzuspielen.	
  		\item[/F410/]Kombinierbarkeit \newline
			Die Punkte /F10/ bis /F330/ sollen im Rahmen der Rechnerleistung beliebig kombinierbar sein.
        \end{description}

\subsection{Wunschkriterien}
	\begin{description}
		\item[/F500/]Plugins \newline
			Das Programm soll durch Plugins um weitere Knotentypen erweiterbar sein.
		\item[/F510/]Quantisieren der Farbpalette eines Videos \newline
			Die Farbpalette eines Videos soll quantisiert werden können.
	\end{description}

 

