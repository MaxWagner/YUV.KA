\section{Globale Testfälle}

\subsection{Testfälle}

\textbf{Folgende Funktionssequenzen sind zu überprüfen:}

\begin{description}
	\item[/T10/] Pipeline-Konstruktion
		\begin{trivlist}
			\item[--] Der Benutzer startet das Programm.
			\item[--] Der Benutzer wählt ``Neue Pipeline erstellen''.
			\item[--] Der Benutzer erzeugt jede Art von Knoten einmal, indem er jeden Knoten per ``Drag-and-Drop'' aus einer am unteren Bildschirmrand
				angebrachten Leiste erzeugt.
			\item[--] Der Benutzer verbindet die erstellten Knoten miteinander, indem er Kanten zwischen den Aus- und Eingängen der Knoten erstellt.
		\end{trivlist}
	\item[/T20/] Manipulation einer Pipeline
		\begin{trivlist}
			\item[--] Der Benutzer startet das Programm und erstellt eine beliebige Pipeline mit mindestens zwei Knoten, davon mindestens ein Manipulationsknoten mit Optionen 
				und mindestens einer Kante.
			\item[--] Der Benutzer bewegt einen gesetzten Knoten innerhalb der Pipeline per ``Drag-and-Drop''.
			\item[--] Der Benutzer verändert das Ziel einer gesetzten Kante per ``Drag-and-Drop''.
			\item[--] Der Benutzer löscht eine Kante.
			\item[--] Der Benutzer verändert eine Einstellung eines Manipulationsknotens.
			\item[--] Der Benutzer löscht einen Knoten.
		\end{trivlist}
	\item[/T30/] Sicherung einer Pipeline
		\begin{trivlist}
			\item[--] Der Benutzer startet das Programm und erstellt eine beliebige Pipeline.
			\item[--] Der Benutzer speichert die Pipeline.
			\item[--] Der Benutzer wählt ``Neue Pipeline erstellen''.
			\item[--] Der Benutzer lädt die gesicherte Pipeline.
		\end{trivlist}
\newpage
	\item[/T40/] Videoverarbeitung
		\begin{trivlist}
			\item[--] Der Benutzer startet das Programm und erstellt eine beliebige, zusammenhängende, zyklenfreie Pipeline mit mindestens einem Eingabe-, Wiedergabe- sowie 
				Manipulationsknoten mit Optionen.
			\item[--] Der Benutzer ändert die Quelle eines Eingabeknotens.
			\item[--] Der Benutzer öffnet den Wiedergabeknoten und beginnt das Video abzuspielen.
			\item[--] Der Benutzer ändert eine Option eines Manipulationsknotens, während das Video abspielt.
			\item[--] Der Benutzer ändert die Abspielgeschwindigkeit des Videos.
			\item[--] Der Benutzer pausiert die Videowiedergabe.
			\item[--] Der Benutzer setzt die Videowiedergabe fort.
			\item[--] Der Benutzer setzt die Videowiedergabe zurück.
			\item[--] Der Benutzer speichert das manipulierte Video als YUV-Datei.
		\end{trivlist}
	\item[/T50/] Videoanalyse
		\begin{trivlist}
			\item[--] Der Benutzer startet das Programm und erstellt eine beliebige, zusammenhängende, zyklenfreie Pipeline mit mindestens einem Eingabe-, Überlagerungs- sowie 
				Diagrammknoten.
			\item[--] Der Benutzer deaktiviert einen Diagrammknoten.
			\item[--] Der Benutzer öffnet einen Überlagerungsknoten.
			\item[--] Der Benutzer beginnt die Videowiedergabe.
			\item[--] Der Benutzer fügt eine Überlagerungsoption hinzu, während das Video abspielt.
			\item[--] Der Benutzer entfernt die hinzugefügte Überlagerungsoption.
			\item[--] Der Benutzer setzt die Videowiedergabe zurück und schließt den Überlagerungsknoten.
			\item[--] Der Benutzer reaktiviert den deaktivierten Diagrammknoten und öffnet diesen.
			\item[--] Der Benutzer wählt ein Referenzvideo aus und fügt einen Analysegraphen hinzu.
			\item[--] Der Benutzer startet erneut die Videowiedergabe.
			\item[--] Der Benutzer ändert den Typ des Analysegraphen.
			\item[--] Der Benutzer löscht den Analysegraphen.
		\end{trivlist}
\end{description}

\newpage

\subsection{Datenkonsistenzen}

\textbf{Folgende Datenkonsistenzen sind einzuhalten:}

\begin{description}
	\item[/T100/] Ergebnislose Wiedergabe verhindern ~\\
		Wenn der Benutzer keinen Endknoten geöffnet hat, ist die Videowiedergabe nicht anwählbar.
	\item[/T110/] Verarbeitung ohne Eingabe verhindern ~\\
		Falls ein Eingabeknoten ohne gültige Videoquelle mit der Pipeline verbunden ist, ist weder die Videowiedergabe, noch das Speichern von Videos als YUV-Datei möglich.
	\item[/T120/] Strukturbrüche während der Wiedergabe verhindern ~\\
		Während der Videowiedergabe kann der Benutzer die Pipelinestruktur nicht verändern.
	\item[/T130/] Inkonsistenz eines gespeicherten Videos verhindern ~\\
		Während des Speicherns eines Videos als YUV-Datei kann der Benutzer keinerlei Änderungen an der Pipeline vornehmen.
\end{description}
