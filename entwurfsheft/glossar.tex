\section{Glossar}

\begin{description}
    \item[(Video-)Artefakte] Darstellungsfehler, welche sich dadurch zeigen, dass das enkodierte Video sichtlich stark von dem Referenzvideo abweicht
    \item[Asynchrone UI] Nichtblockend gestaltete Oberfläche, die bei Ladeoperationen weiterhin auf Interaktion reagiert

    \item[Differenzvideo] Videostream, welcher aus den Daten besteht, die bei der pixelweisen Subtraktion der Farbwerte zweier Videos entstehen
    \item[Drag-and-Drop] Bedienungsmuster, bei dem virtuelle Objekte mit der linken Maustaste ``gegriffen'' und ``gezogen'' werden

    \item[Farbkanal] Ein Farbbild besteht in der Regel aus 3 Farbkanälen, beispielsweise RGB (rot, grün, blau) oder \emph{YUV}.
    \item[Frame] Einzelbild eines Videos, bei \emph{H.264} unterteilt in \emph{Macroblöcke}

    \item[H.264] Weit verbreiteter \emph{Video-Codec} und Ziel-Codec der zu testenden Encoder
    \item[Histogramm] Diagrammart ähnlich einem Balkendiagramm, welche aber auch den Flächeninhalt eines Balken korrekt skaliert darstellt

    \item[Interface, UI] Benutzeroberfläche. Man unterscheidet zwischen Konsolenoberfläche und graphischer Oberfläche.
    \item[Inter-Frame] Videoframe, dessen Daten aus einem oder mehreren benachbarten Frames berechnet werden
    \item[Intra-Frame] Videoframe, dessen Daten unabhängig von denen anderer Frames gespeichert sind. Vergleiche \emph{Inter-Frame}.

    \item[Macroblock] Bei \emph{H.264} 16x16 Pixel große Blöcke, in die jeder \emph{Frame} aufgeteilt wird. Im Gegensatz zu älteren Codecs kann bei H.264 für jeden Macroblock eine unterschiedliche \emph{Inter/Intra-Frame}-Entscheidung getroffen werden. 
    \item[Move Vector] Vektor, um den jeder Referenzframe eines \emph{Inter-Frames} zusätzlich verschoben werden kann. Ermöglicht eine effiziente Kodierung von verschiebenden Bewegungen.
    \item[Multithreading] Programmiertechnik, in der die zu verrichtende Arbeit auf mehrere Prozessfäden verteilt wird. Dies ermöglicht Parallelismus und die optimale Auslastung von Multikern-Systemen

    \item[Noise] Ein randomisiertes Signal. Allgemein als ``Bildrauschen'' zu verstehen.

    \item[Overlay] Ein transparent über ein anderes Objekt gezeichnetes Objekt
    \item[Peak signal-to-noise ratio] Maß für den wahrgenommenen Qualitätsverlust eines komprimierten Videoframes
    \item[Pipeline] Hintereinanderschaltung von verschiedenen Knotentypen zum Einlesen, Modifizieren und Analysieren von Videos, in der UI dargestellt als Graph mit über \emph{Drag-and-Drop} erstellbaren Kanten

    \item[Resolution] Bildauflösung

    \item[Tick] Zeitabschnitt zur Berechnung aller Manipulationen und Analysen eines bestimmten Frames.

    \item[Undo/Redo] ``Aktion zurücknehmen/wiederholen''

    \item[Video-Codec] Spezifikation eines Verfahrens, um Videodaten (meist verlustbehaftet) zu komprimieren
    \item[Videoencoder] Programm, welches rohe Videodaten als Eingabe annimmt und in einem bestimmten \emph{Video-Codec} entsprechendes Format bringt

    \item[YUV] Eine Familie von Farbpaletten, welche die Sehart des menschlichen Auges in Betracht nehmen
\end{description}
