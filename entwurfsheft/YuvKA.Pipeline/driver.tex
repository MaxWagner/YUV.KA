\subsection{Pipeline Driver}

\subsubsection{YuvKA.Pipeline.PipelineDriver}

\begin{verbatim}
public static class PipelineDriver
\end{verbatim}

\paragraph{Beschreibung}~\\
Die Klasse \name{PipelineDriver} ist für die Ausführung der Pipeline zuständig. Sie löst die gegenseitigen Abhängigkeiten der \name{Node}-Instanzen auf, um dann in Reihenfolge einer topologischer Sortierung deren \name{ProcessFrame}-Methoden aufrufen zu können.
Die Klasse soll möglichst parallel implementiert werden, sodass mehrere aufeinanderfolgende Frames gleichzeitig berechnet werden.

\paragraph{Typmember}
\begin{itemize}

\property{RenderFrames}
	\begin{verbatim}
public static IObservable<IDictionary<Node.Output, Frame>> RenderFrames(
    IEnumerable<Node> startNodes, int frameIndex, CancellationToken token)
    \end{verbatim}
	Berechnet die Pipeline ab dem angegebenen Frame-Index und liefert eine asynchrone Aufzählung zurück, die pro Frame-Index ein Dictionary enthält, das jedem Ausgang der gegebenen Knoten \name{startNodes} den jeweils berechneten Frame zuordnet.

    Parallelitätszusicherungen: Die Methode führt auf einer \name{Node}-Instanz \name{ProcessFrame} immer seriell mit strikt aufsteigendem Frame-Index aus. Die gleiche Ordnung gilt für den Rückgabewert. Die Berechnung wird bei Aktivierung des \name{token}s am nächstmöglichen Punkt beendet.

\end{itemize}
